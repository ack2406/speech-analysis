Uzyskane wyniki wskazują na różnice w pasmach częstotliwości między językami, co potwierdza teorię Tomatisa. W szczególności dominujące zakresy częstotliwości dla języka angielskiego (2-14 kHz) oraz niemieckiego (125-3500 Hz) są zgodne z wcześniejszymi badaniami. Zastosowane metody pozwoliły na precyzyjną ekstrakcję cech akustycznych i porównanie wyników dla różnych języków.

Możliwości rozszerzenia tych badań obejmują analizę innych cech mowy, takich jak intonacja czy rytm, które również mogą charakteryzować dany język. Ponadto, badania mogą być rozszerzone na większą liczbę języków, aby potwierdzić ogólność wyników.
